\documentclass{scrartcl}

\usepackage[utf8]{inputenc}

\title{Beschreibung: Client für die Datenbank AniDB}
\author{Daniel Alt\\Robert Edelmann\\Denis Moskal\\Jonas Weber}
\date{3. April 2012}

\begin{document}
\maketitle

\section*{Beschreibung}

Client für die Onlinedatenbank AniDB.
Der Client verbindet sich per UDP API mit der Datenbank. Das Programm hasht lokale Video-Dateien und 
vergleicht diese mit den online verfügbaren Informationen. Werden Treffer gefunden, lädt das Programm
zusätzliche Informationen von AniDB und stellt diese übersichtlich dar. Das Programm selbst wird die Video-Dateien mit 
den vorhandenen Daten und vorgegebenen Patterns umbenennen und verschieben.
Diese Informationen werden aus Caching-Gründen auch in einer lokalen Datenbank vorgehalten.

\section*{Verwendete Technologien}

Als Technologien werden wie gefordert das Windows Presentation Framework, das Entity Framework und Multithreading
verwendet.

\subsection*{Einschränkungen der AniDB-API}

Für die Verwendung der AniDB-API als Free-User gibt es einige Einschränkungen: Es darf zum Beispiel
nur maximal ein Paket pro zwei Sekunden versendet werden. Dies wird durch eingebaute Mechanismen erreicht.

\section*{Relevanz / Anwendunsgebiete}

Zwei Leute aus unserer Gruppe sind absolute Freaks und brauchen eine solche Software zur Verwaltung
ihrer riesigen Serienbestände, die sie eh nie angucken können.

\end{document}
